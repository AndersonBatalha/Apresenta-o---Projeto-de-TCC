\documentclass{beamer}
\usepackage{amsmath}
\usepackage[english,brazil]{babel}
\usepackage[brazil]{babel}
\usepackage[utf8x]{inputenc}
\usepackage{graphicx}
\usepackage{url,color,ae}
\usepackage{listings,color,upquote}
\usepackage[T1]{fontenc}
\usepackage{graphicx}
\usepackage{amssymb}
\usepackage{amsmath}
\usepackage{color,colortbl,multirow}
\hypersetup{pdftitle={IFC - Campus Araquari},
    pdfsubject={Informatica}
}
\usepackage{url}
\usetheme[numbers,totalnumber,compress]{Madrid}
% \usetheme{Warsaw}
% \usetheme{Boadilla}
% \usetheme{CambridgeUS}
% \usetheme{Montpellier}
% \usetheme{Hannover}
%\usetheme{Dresden}

\definecolor{verdeescuro}{rgb}{0,0.40,0}
\definecolor{verde}{rgb}{0.55,0.78,0.25}

% Comando: \shell
\newcommand{\shell}
{\lstset{language=csh,basicstyle=\ttfamily\footnotesize,tabsize=3,frame=single,showtabs=false,showspaces=false,firstnumber=last,numbers=left,numberstyle=\tiny,linewidth=0.98\linewidth,xleftmargin=21pt,tab=$\to$,float=tbph,extendedchars,breaklines,showstringspaces=false,identifierstyle=\color{colIdentifier},keywordstyle=\color{colKeys},stringstyle=\color{colString},commentstyle=\color{colComments},backgroundcolor=\color{hellgelb},columns=flexible,captionpos=b,aboveskip=\bigskipamount}}

% ------------------------------------------------------------------ %
% Deixando o tema mais verde. Comente a linha abaixo se não gostar %
%\setbeamercolor{structure}{fg=verdeescuro}
% ------------------------------------------------------------------ %
% Deixando o verde mais claro para combinar com o logo do IFET
% Descomente as linhas abaixo

\setbeamercolor{structure}{fg=verde}
\setbeamercolor{title}{fg=black,bg=verde!80!black}
\setbeamercolor{frametitle}{fg=black,bg=verde!25}
\setbeamercolor{block body}{fg=black,bg=verde!15}
\setbeamercolor{block title}{fg=white,bg=verdeescuro}
\setbeamercolor{item}{fg=verde!80!black} %cor marcador
\setbeamertemplate{items}[triangle] % tipos de marcadores square, triangle, circle,ball
% ------------------------------------------------------------------ %
% para não aparecer aqueles ícones de navegação no canto direito inferior
%\beamertemplatenavigationsymbolsempty

\title[Sistemas de Informação]{\textbf{Análise de sentimentos\\ relacionados a notícias}} %Título Apresentação
\subtitle{\textit{TCC I}}

\institute[IFC]{
Instituto Federal Catarinense -- IFC\\
Campus Araquari \\
}

\author{Anderson Pontes Batalha}

% ----- Logo IFC --------%
\pgfdeclareimage[width=1.8cm]{logo}{logoifccompleto}
\logo{\pgfuseimage{logo}}

\date{\tiny{\today}}

% ------------ Inicio do documento ---------------%

\begin{document}

\begin{frame}
    \maketitle
\end{frame}

\AtBeginSection[]{
  \frame[allowframebreaks]{
    \frametitle{}
    \tableofcontents[current,currentsection]
  }
}

% -------------------------------------------------%
\section{Ideia central}
\setbeamercovered{transparent}%efeito de transparência para transição uma vez colocada na apresentação aplica a todos os frames
\begin{frame}%início de frame, cada frame é uma tela da apresentação%
\frametitle{\textbf{Definição da Linguagem}\transdissolve}%efeito de transição do frame%
\begin{itemize}%início da lista%
\item<1-> Selecionar notícias de grande relevância
%\item<2-> 
%\item<3-> 
%\item<4-> 
\end{itemize}
\end{frame}

\section{Motivação}
\begin{frame}%início de frame, cada frame é uma tela da apresentação%
\frametitle{\textbf{Estrutura Básica}\transdissolve}%efeito de transição do frame%
\begin{itemize}%início da lista%
\item<1-> O documento possui uma estrutura de árvore onde existe uma relação de parentesco entre os elementos.
\item<2-> Separação entre o conteúdo e a formatação do documento.
\end{itemize}
\end{frame}

\section{Principais Tags}
\begin{frame}%início de frame, cada frame é uma tela da apresentação%
\frametitle{\textbf{Principais Tags}\transdissolve}%efeito de transição do frame%
\begin{itemize}%início da lista%
\item<1-> Não existem tags definidas, por ser uma linguagem extensiva, permite a definição de um número infinito de tags, ficando a critério do usuário.
\end{itemize}
\end{frame}

\section{Funcionamento}
\begin{frame}[fragile]%início de frame, cada frame é uma tela da apresentação%
\frametitle{\textbf{Funcionamento}\transdissolve}%efeito de transição do frame%
\begin{itemize}%início da lista%
\item<1-> Um documento XML inicia informando a versão do XML e a codificação dos caracteres contidos. Todo documento precisa ter esta linha inicial.
\begin{verbatim}
<?xml version="1.0" encoding="utf-8"?>
\end{verbatim}
\item<2-> Em seguida é criada uma “tag” que engloba todo o documento.
\begin{verbatim}
<elemento_raiz> 

</elemento_raiz>	
\end{verbatim}
\item<3-> A criação de outros elementos segue o mesmo padrão, abre a tag, insere o conteúdo e fecha a tag. Exemplo:
\begin{verbatim}
<elementopai>
  	<elementofilho>Informação</elementofilho>
</elementopai>
\end{verbatim}
\end{itemize}
\end{frame}

\section{Modelo de utilização}
\begin{frame}%início de frame, cada frame é uma tela da apresentação%
\frametitle{\textbf{Modelo de utilização}\transdissolve}%efeito de transição do frame%
\begin{verbatim}%início da lista%
<?xml version="1.0" encoding="UTF-8"?>

<produtos>

    <produto>

        <codigo>1</codigo>

        <descricao>Livro</descricao>

        <preco>21.75</preco>

    </produto>

    <produto>

        <codigo>2</codigo>

        <descricao>Tênis</descricao>

        <preco>99.99</preco>
        
    </produto>

</produtos>
\end{verbatim}
\end{frame}


\begin{frame}
\frametitle{Referência Bibliográfica}
{\scriptsize http://goo.gl/TPIbKG} \\
{\scriptsize http://goo.gl/12G6Aw} \\
{\scriptsize http://goo.gl/koFHNU2} \\
{\scriptsize http://goo.gl/GDw5gN} \\
{\scriptsize http://goo.gl/nrEzzF} \\
{\scriptsize http://goo.gl/S6Ylac} \\
{\scriptsize http://goo.gl/VEBz6A} \\
{\scriptsize http://goo.gl/iSpnOy} \\
\end{frame}

\end{document}