\documentclass{beamer}
\usepackage{amsmath}
\usepackage[english,brazil]{babel}
\usepackage[brazil]{babel}
\usepackage[utf8]{inputenc}
\usepackage{graphicx}
\usepackage{url,color,ae}
\usepackage{listings,color,upquote}
\usepackage[T1]{fontenc}
\usepackage{graphicx}
\usepackage{amssymb}
\usepackage{amsmath}
\usepackage{indentfirst}
\usepackage{color,colortbl,multirow}
\usepackage{biblatex}
\hypersetup{pdftitle={IFC - Campus Araquari},
    pdfsubject={Informatica}
}
\usepackage{url}
\usetheme[numbers,totalnumber,compress]{Madrid}
% \usetheme{Warsaw}
% \usetheme{Boadilla}
% \usetheme{CambridgeUS}
% \usetheme{Montpellier}
% \usetheme{Hannover}
%\usetheme{Dresden}

\definecolor{verdeescuro}{rgb}{0,0.40,0}
\definecolor{verde}{rgb}{0.55,0.78,0.25}

% Comando: \shell
\newcommand{\shell}
{\lstset{language=csh,basicstyle=\ttfamily\footnotesize,tabsize=3,frame=single,showtabs=false,showspaces=false,firstnumber=last,numbers=left,numberstyle=\tiny,linewidth=0.98\linewidth,xleftmargin=21pt,tab=$\to$,float=tbph,extendedchars,breaklines,showstringspaces=false,identifierstyle=\color{colIdentifier},keywordstyle=\color{colKeys},stringstyle=\color{colString},commentstyle=\color{colComments},backgroundcolor=\color{hellgelb},columns=flexible,captionpos=b,aboveskip=\bigskipamount}}

% ------------------------------------------------------------------ %
% Deixando o tema mais verde. Comente a linha abaixo se não gostar %
\setbeamercolor{structure}{fg=verdeescuro}
% ------------------------------------------------------------------ %
% Deixando o verde mais claro para combinar com o logo do IFET
% Descomente as linhas abaixo

\setbeamercolor{structure}{fg=verde}
\setbeamercolor{title}{fg=black,bg=verde!80!black}
\setbeamercolor{frametitle}{fg=black,bg=verde!25}
\setbeamercolor{block body}{fg=black,bg=verde!15}
\setbeamercolor{block title}{fg=white,bg=verdeescuro}
\setbeamercolor{item}{fg=verde!80!black} %cor marcador
\setbeamertemplate{items}[triangle] % tipos de marcadores square, triangle, circle,ball
% ------------------------------------------------------------------ %
% para não aparecer aqueles ícones de navegação no canto direito inferior
%\beamertemplatenavigationsymbolsempty

\title[Sistemas de Informação]{\textbf{Análise de sentimentos\\ relacionados a notícias}} %Título Apresentação
\subtitle{\textit{TCC I}}

\institute[IFC]{
Instituto Federal Catarinense -- IFC\\
Campus Araquari \\
}

\author{Anderson Pontes Batalha}

% ----- Logo IFC --------%
\pgfdeclareimage[width=1.8cm]{logo}{logoifccompleto}
\logo{\pgfuseimage{logo}}

\date{\tiny{\today}}

% ------------ Inicio do documento ---------------%
\begin{document}

\begin{frame}
    \maketitle
\end{frame}

\AtBeginSection[]{
  \frame[allowframebreaks]{
    \frametitle{}
    \tableofcontents[current,currentsection]
  }
}

% -------------------------------------------------%
\section{Tema}
\setbeamercovered{transparent}%efeito de transparência para transição uma vez colocada na apresentação aplica a todos os frames
\begin{frame}%início de frame, cada frame é uma tela da apresentação%
\frametitle{\textbf{Tema}\transdissolve}%efeito de transição do frame%
\begin{itemize}%início da lista%
\item<1-> Utilizar a análise de sentimentos para determinar a polaridade de mensagens relacionadas a fatos de grande repercussão nas redes sociais (Twitter e Facebook);
\end{itemize}
\end{frame}

\section{Definição}
\setbeamercovered{transparent}%efeito de transparência para transição uma vez colocada na apresentação aplica a todos os frames
\begin{frame}%início de frame, cada frame é uma tela da apresentação%
\frametitle{\textbf{Definição}\transdissolve}%efeito de transição do frame%
\begin{itemize}%início da lista%
\item<1-> "Análise de sentimentos ou mineração de opinião é o estudo computacional de opiniões, sentimentos e emoções expressas em texto."
\end{itemize}
\end{frame}

\section{Justificativa}
\begin{frame}%início de frame, cada frame é uma tela da apresentação%
\frametitle{\textbf{Justificativa}\transdissolve}%efeito de transição do frame%
\begin{itemize}%início da lista%
\item<1->Popularização das redes sociais, por consequência ocorreu o aumento do tráfego de dados gerados pelas redes sociais
\item<3->Usuários estão constantemente trocando informações entre si e emitindo opiniões sobre os mais variados assuntos
\item<4->Comportamento das pessoas é influenciado pelas opiniões de outras pessoas
\end{itemize}

\end{frame}

\section{Aplicações}
\begin{frame}%início de frame, cada frame é uma tela da apresentação%
\frametitle{\textbf{Aplicações}\transdissolve}%efeito de transição do frame%
\begin{itemize}%início da lista%
\item<1->Comercial
\item<2->Detecção de spam
\item<3->Reviews de filmes, produtos ou serviços
\item<4->Mercado financeiro
\item<5->Notícias
\item<6->Política

\end{itemize}
\end{frame}

\section{Conceitos básicos}
\begin{frame}%início de frame, cada frame é uma tela da apresentação%
\frametitle{\textbf{Tipos de opinião}\transdissolve}%efeito de transição do frame%
\begin{itemize}%início da lista%
\item<1->Opinião regular e opinião comparativa\\
\qquad"Aquele é um bom restaurante"\\
\qquad"O restaurante A é melhor que o restaurante B"\\
\item<2->Opinião direta e opinião indireta\\
\qquad "Este remédio é muito bom"\\
\qquad "Minha gripe piorou depois que tomei este remédio"\\
\item<3->Opinião implícita e opinião explícita\\
\qquad "A câmera que comprei parou de funcionar em menos de um mês"\\
\end{itemize}

\end{frame}

\begin{frame}%início de frame, cada frame é uma tela da apresentação%
\frametitle{\textbf{Conceitos básicos}\transdissolve}%efeito de transição do frame%
\begin{itemize}%início da lista%
\item<1->Fato e Opinião
\item<2->Entidade
\item<3->Aspecto
\item<4->Polaridade
\item<5->Emoção
\end{itemize}
\end{frame}

\section{Metodologia}

\begin{frame}%início de frame, cada frame é uma tela da apresentação%
\frametitle{\textbf{Escolha das notícias}\transdissolve}%efeito de transição do frame%
\begin{itemize}
    \item Selecionar fatos de grande repercussão, de preferência recentes, para servir de base para a coleta dos dados
\end{itemize}
\end{frame}

\begin{frame}%início de frame, cada frame é uma tela da apresentação%
\frametitle{\textbf{Coleta dos dados}\transdissolve}%efeito de transição do frame%
\begin{itemize}%início da lista%
\item<1->Twitter\footnote[1]{https://developer.twitter.com/en/docs/api-reference-index} e Facebook\footnote[2]{https://developers.facebook.com/docs/graph-api?locale=pt\_BR} fornecem APIs para acesso aos dados públicos dos usuários
\item<2->Bibliotecas permitem construir aplicações que acessem esses dados
\end{itemize}
\end{frame}

\begin{frame}%início de frame, cada frame é uma tela da apresentação%
\frametitle{\textbf{Limpeza dos dados}\transdissolve}%efeito de transição do frame%
\begin{itemize}
\item Remoção de Stop words\footnote[1]{https://www.ranks.nl/stopwords}
\item Stemming
\end{itemize}
\end{frame}


\begin{frame}%início de frame, cada frame é uma tela da apresentação%
\frametitle{\textbf{Aplicação dos métodos de análise de sentimentos}\transdissolve}%efeito de transição do frame%
\begin{itemize}
    \item Aplicação dos métodos de análise de sentimentos no conjunto de dados
    \item Métodos utilizados\\
        \qquad Opinion Lexicon, SentiWordNet, SenticNet, PANAS-t, OpinionFinder e Vader (disponíveis apenas em inglês)\\
        \qquad SentiLex-PT e OpLexicon (disponíveis em
        português)\\
\end{itemize}
\end{frame}

\begin{frame}%início de frame, cada frame é uma tela da apresentação%
\frametitle{\textbf{Abordagens de análise de sentimentos}\transdissolve}%efeito de transição do frame%
\begin{itemize}%início da lista%
\item<1->Baseada em léxico
\item<2->Baseada em aprendizado de máquina
\end{itemize}
\end{frame}

\begin{frame}%início de frame, cada frame é uma tela da apresentação%
\frametitle{\textbf{Níveis de análise}\transdissolve}%efeito de transição do frame%
\begin{itemize}%início da lista%
\item<1->Documento
\item<2->Sentença
\item<3->Entidade
\item<4->Palavra
\end{itemize}
\end{frame}

\begin{frame}%início de frame, cada frame é uma tela da apresentação%
\frametitle{\textbf{Resultados}\transdissolve}%efeito de transição do frame%
\begin{itemize}
    \item Estabelecer uma comparação, com base nos resultados obtidos, de cada um métodos, apontando suas principais características, pontos fortes e fracos.
\end{itemize}
\end{frame}

\section{Objetivo geral}
\begin{frame}%início de frame, cada frame é uma tela da apresentação%
\frametitle{\textbf{Objetivo geral}\transdissolve}%efeito de transição do frame%
\begin{itemize}%início da lista%
\item<1->Aplicar os métodos de detecção de sentimentos e definir qual obteve melhor desempenho utilizando os dados coletados.
\end{itemize}
\end{frame}

\section{Objetivos específicos}
\begin{frame}%início de frame, cada frame é uma tela da apresentação%
\frametitle{\textbf{Objetivos específicos}\transdissolve}%efeito de transição do frame%
\begin{itemize}%início da lista%
\item<1->Selecionar as notícias que serão utilizadas como referência para a coleta de
dados.
\item<2->Pesquisar em artigos relacionados os principais métodos de detecção de sentimentos.
\item<3->Definir um método para coleta dos dados, e por quanto tempo será realizada.
\item<4->Realizar a limpeza e pré-processamento dos dados, retirando palavras que não exprimem sentimentos (stop words), além de gírias e abreviações.
\item<5->Definir quais métricas para avaliação dos algoritmos de análise de sentimentos.
\item<6->Utilizar a abordagem léxica. Este método se apresenta mais eficiente em comparação com os métodos supervisionados. 

\end{itemize}
\end{frame}

\section{Desafios}
\begin{frame}%início de frame, cada frame é uma tela da apresentação%
\frametitle{\textbf{Desafios}\transdissolve}%efeito de transição do frame%
\begin{itemize}%início da lista%
\item<1->Linguagem informal (gírias, abreviações)
\item<2->Ironia e sarcasmo
\item<3->Algumas palavras podem depender de um contexto para serem compreendidas de maneira correta
\item<4->A grande maioria dos métodos disponíveis se baseiam em textos em inglês.

\end{itemize}
\end{frame}

\begin{frame}

\frametitle{Referências Bibliográficas}
BENEVENUTO, Fabrício; RIBEIRO, Filipe; ARAÚJO, Matheus. Métodos para Análise de
Sentimentos em mídias sociais. 2015.\newline

BECKER, Karin; TUMITAN, Diego. Introdução à mineração de opiniões: Conceitos, aplicações e desafios. Simpósio brasileiro de banco de dados, 2013.\newline

SILVA, Nadia Felix Felipe da. Análise de sentimentos em textos curtos provenientes de redes sociais. 2016. Tese de Doutorado. Universidade de São Paulo.\newline

LIU, Bing. Sentiment Analysis and Subjectivity. Handbook of natural language processing, v. 2, p. 627-666, 2010.

\end{frame}

\end{document}


