\documentclass{beamer}
\usepackage{amsmath}
\usepackage[english,brazil]{babel}
\usepackage[brazil]{babel}
\usepackage[utf8x]{inputenc}
\usepackage{graphicx}
\usepackage{url,color,ae}
\usepackage{listings,color,upquote}
\usepackage[T1]{fontenc}
\usepackage{graphicx}
\usepackage{amssymb}
\usepackage{amsmath}
\usepackage{indentfirst}
\usepackage{color,colortbl,multirow}
\hypersetup{pdftitle={IFC - Campus Araquari},
    pdfsubject={Informatica}
}
\usepackage{url}
\usetheme[numbers,totalnumber,compress]{Madrid}
% \usetheme{Warsaw}
% \usetheme{Boadilla}
% \usetheme{CambridgeUS}
% \usetheme{Montpellier}
% \usetheme{Hannover}
%\usetheme{Dresden}

\definecolor{verdeescuro}{rgb}{0,0.40,0}
\definecolor{verde}{rgb}{0.55,0.78,0.25}

% Comando: \shell
\newcommand{\shell}
{\lstset{language=csh,basicstyle=\ttfamily\footnotesize,tabsize=3,frame=single,showtabs=false,showspaces=false,firstnumber=last,numbers=left,numberstyle=\tiny,linewidth=0.98\linewidth,xleftmargin=21pt,tab=$\to$,float=tbph,extendedchars,breaklines,showstringspaces=false,identifierstyle=\color{colIdentifier},keywordstyle=\color{colKeys},stringstyle=\color{colString},commentstyle=\color{colComments},backgroundcolor=\color{hellgelb},columns=flexible,captionpos=b,aboveskip=\bigskipamount}}

% ------------------------------------------------------------------ %
% Deixando o tema mais verde. Comente a linha abaixo se não gostar %
\setbeamercolor{structure}{fg=verdeescuro}
% ------------------------------------------------------------------ %
% Deixando o verde mais claro para combinar com o logo do IFET
% Descomente as linhas abaixo

\setbeamercolor{structure}{fg=verde}
\setbeamercolor{title}{fg=black,bg=verde!80!black}
\setbeamercolor{frametitle}{fg=black,bg=verde!25}
\setbeamercolor{block body}{fg=black,bg=verde!15}
\setbeamercolor{block title}{fg=white,bg=verdeescuro}
\setbeamercolor{item}{fg=verde!80!black} %cor marcador
\setbeamertemplate{items}[triangle] % tipos de marcadores square, triangle, circle,ball
% ------------------------------------------------------------------ %
% para não aparecer aqueles ícones de navegação no canto direito inferior
%\beamertemplatenavigationsymbolsempty

\title[Sistemas de Informação]{\textbf{Análise de sentimentos\\ relacionados a notícias}} %Título Apresentação
\subtitle{\textit{TCC I}}

\institute[IFC]{
Instituto Federal Catarinense -- IFC\\
Campus Araquari \\
}

\author{Anderson Pontes Batalha}

% ----- Logo IFC --------%
\pgfdeclareimage[width=1.8cm]{logo}{logoifccompleto}
\logo{\pgfuseimage{logo}}

\date{\tiny{\today}}

% ------------ Inicio do documento ---------------%

\begin{document}

\begin{frame}
    \maketitle
\end{frame}

\AtBeginSection[]{
  \frame[allowframebreaks]{
    \frametitle{}
    \tableofcontents[current,currentsection]
  }
}

% -------------------------------------------------%
\section{Tema}
\setbeamercovered{transparent}%efeito de transparência para transição uma vez colocada na apresentação aplica a todos os frames
\begin{frame}%início de frame, cada frame é uma tela da apresentação%
\frametitle{\textbf{Tema}\transdissolve}%efeito de transição do frame%
\begin{itemize}%início da lista%
\item<1-> Utilizar a análise de sentimentos para determinar a polaridade de mensagens relacionadas a fatos de grande repercussão nas redes sociais (Twitter e Facebook);
\end{itemize}
\end{frame}

\section{Delimitação do tema}
\begin{frame}%início de frame, cada frame é uma tela da apresentação%
\frametitle{\textbf{Delimitação do tema}\transdissolve}%efeito de transição do frame%
\begin{itemize}%início da lista%
\item<1->
\end{itemize}
\end{frame}

\section{Motivação}
\begin{frame}%início de frame, cada frame é uma tela da apresentação%
\frametitle{\textbf{Motivação}\transdissolve}%efeito de transição do frame%
\begin{itemize}%início da lista%
\item<1->Existem poucos trabalhos com foco em opiniões expressas sobre notícias.
\end{itemize}

\end{frame}

\section{Justificativa}
\begin{frame}%início de frame, cada frame é uma tela da apresentação%
\frametitle{\textbf{Justificativa}\transdissolve}%efeito de transição do frame%
\begin{itemize}%início da lista%
\item<1->Popularização das redes sociais
\item<2->Grandes quantidades de dados gerados
\item<3->Usuários estão constantemente trocando informações entre si e emitindo opiniões sobre os mais variados assuntos
\item<4->Comportamento das pessoas é influenciado pelas opiniões de outras pessoas
\end{itemize}

\end{frame}

\section{Objetivo geral}
\begin{frame}%início de frame, cada frame é uma tela da apresentação%
\frametitle{\textbf{Objetivo geral}\transdissolve}%efeito de transição do frame%
\begin{itemize}%início da lista%
\item<1->Aplicar os métodos de detecção de sentimentos e definir qual obteve melhor desempenho utilizando os dados coletados.
\end{itemize}

\end{frame}

\section{Objetivos específicos}
\begin{frame}%início de frame, cada frame é uma tela da apresentação%
\frametitle{\textbf{Objetivos específicos}\transdissolve}%efeito de transição do frame%
\begin{itemize}%início da lista%
\item<1->Selecionar as notícias que serão utilizadas como referência para a coleta de
dados.
\item<2->Pesquisar em artigos relacionados os principais métodos de detecção de sentimentos.
\item<3->Definir um método para coleta dos dados, e por quanto tempo será realizada.
\item<4->Realizar a limpeza e pré-processamento dos dados, retirando palavras que não exprimem sentimentos (stop words), além de gírias e abreviações.
\item<5->Definir quais métricas para avaliação dos algoritmos de análise de sentimentos.
\item<6->Utilizar aprendizado não supervisionado. (dicionários léxicos)
\end{itemize}

\end{frame}

\section{Metodologia}
\begin{frame}%início de frame, cada frame é uma tela da apresentação%
\frametitle{\textbf{Metodologia}\transdissolve}%efeito de transição do frame%
\begin{itemize}%início da lista%
\item<1->Escolha das notícias
\item<2->Coleta dos dados
\item<3->Limpeza dos dados
\item<4->Aplicação dos métodos de análise de sentimentos
\item<5->Resultados
\end{itemize}
\end{frame}

\begin{frame}%início de frame, cada frame é uma tela da apresentação%
\frametitle{\textbf{Coleta dos dados}\transdissolve}%efeito de transição do frame%
\begin{itemize}%início da lista%
\item<1->Utilização de APIs do Twitter e Facebook 
\item<2->\href{https://developer.twitter.com/en/docs/api-reference-index}{API do Twitter}
\item<3->\href{https://developers.facebook.com/docs/graph-api?locale=pt_BR}{Graph API}
\end{itemize}
\end{frame}

\begin{frame}%início de frame, cada frame é uma tela da apresentação%
\frametitle{\textbf{Métodos supervisionados x não supervisionados}\transdissolve}%efeito de transição do frame%
\begin{itemize}%início da lista%
\item<1->Supervisionado\\
\quad Dados rotulados\\
\quad Definição das classes\\
\quad Dados de treino e teste\\
\quad Novos dados tem classes definidas a partir da similaridade com os dados de treino
\end{itemize}
\end{frame}

\begin{frame}%início de frame, cada frame é uma tela da apresentação%
\frametitle{\textbf{Métodos supervisionados x não supervisionados}\transdissolve}%efeito de transição do frame%
\begin{itemize}%início da lista%
\item<1->Não supervisionado\\
\end{itemize}
\end{frame}

\begin{frame}%início de frame, cada frame é uma tela da apresentação%
\frametitle{\textbf{Níveis de análise}\transdissolve}%efeito de transição do frame%
\begin{itemize}%início da lista%
\item<1->Documento
\item<2->Sentença
\item<3->Entidade
\end{itemize}
\end{frame}


\section{Desafios}
\begin{frame}%início de frame, cada frame é uma tela da apresentação%
\frametitle{\textbf{Desafios}\transdissolve}%efeito de transição do frame%
\begin{itemize}%início da lista%
\item<1->Linguagem informal (gírias, abreviações)
\item<2->Ironia e sarcasmo
\item<3->Idioma: a grande maioria dos métodos disponíveis se baseiam em textos em inglês.

\end{itemize}
\end{frame}

\begin{frame}
\frametitle{Referências Bibliográficas}
\bibliography{refs}
\end{frame}


\end{document}

